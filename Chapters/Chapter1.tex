\chapter{Introduction et Contexte}

\section{Présentation de l'entreprise}
L'Office Chérifien des Phosphates (OCP) est une entreprise marocaine leader dans l'industrie des phosphates, opérant dans l'extraction, la transformation et la commercialisation de phosphates et de produits dérivés. Fondée en 1920, OCP est aujourd'hui le premier exportateur mondial de phosphates et joue un rôle crucial dans l'économie marocaine.

L'entreprise opère sur plusieurs sites industriels répartis sur le territoire marocain, chacun nécessitant une gestion continue des équipes et des astreintes pour assurer la continuité des opérations 24h/24 et 7j/7. Cette exigence opérationnelle permanente nécessite une organisation rigoureuse des plannings et une gestion efficace des ressources humaines.

\section{Contexte du projet}
\subsection{Problématique initiale}
Avant l'implémentation de ce système, la gestion des astreintes et des plannings des équipes était effectuée de manière manuelle, principalement via des tableurs Excel et des documents papier. Cette approche présentait plusieurs limitations :

\begin{itemize}
    \item \textbf{Manque de traçabilité} : Difficulté à suivre l'historique des astreintes et des modifications de planning
    \item \textbf{Erreurs humaines} : Risque élevé d'erreurs dans la saisie et la transmission des informations
    \item \textbf{Manque de synchronisation} : Difficulté à maintenir la cohérence entre les différents sites et équipes
    \item \textbf{Temps de traitement} : Processus long et fastidieux pour la création et la modification des plannings
    \item \textbf{Accès limité} : Les informations n'étaient pas facilement accessibles à tous les acteurs concernés
\end{itemize}

\subsection{Objectifs du projet}
Le projet de développement du système de gestion des astreintes OCP avait pour objectifs principaux :

\begin{enumerate}
    \item \textbf{Digitaliser le processus} : Remplacer les processus manuels par une solution informatique centralisée
    \item \textbf{Améliorer l'efficacité} : Réduire le temps de création et de modification des plannings
    \item \textbf{Assurer la traçabilité} : Garder un historique complet de toutes les modifications et astreintes
    \item \textbf{Faciliter l'accès} : Permettre un accès en temps réel aux informations pour tous les utilisateurs autorisés
    \item \textbf{Standardiser les processus} : Mettre en place des procédures uniformes sur tous les sites
\end{enumerate}

\section{Enjeux et défis}
\subsection{Enjeux opérationnels}
La gestion des astreintes dans un contexte industriel présente des enjeux majeurs :

\begin{itemize}
    \item \textbf{Continuité de service} : Assurer la disponibilité permanente des équipes pour maintenir la production
    \item \textbf{Sécurité} : Respecter les réglementations en matière de temps de travail et de repos
    \item \textbf{Équité} : Répartir équitablement les astreintes entre les membres des équipes
    \item \textbf{Flexibilité} : S'adapter aux contraintes opérationnelles et aux imprévus
\end{itemize}

\subsection{Défis techniques}
Le développement de ce système présentait plusieurs défis techniques :

\begin{itemize}
    \item \textbf{Architecture distribuée} : Gérer plusieurs sites géographiquement séparés
    \item \textbf{Sécurité} : Assurer la confidentialité et l'intégrité des données sensibles
    \item \textbf{Performance} : Garantir des temps de réponse rapides même avec de nombreux utilisateurs
    \item \textbf{Évolutivité} : Concevoir un système capable de s'adapter aux évolutions futures
\end{itemize}

\section{Structure du rapport}
Ce rapport est organisé en cinq chapitres principaux :

\begin{description}
    \item[Chapitre 1] Introduction et contexte du projet
    \item[Chapitre 2] Analyse des besoins et conception du système
    \item[Chapitre 3] Implémentation technique et développement
    \item[Chapitre 4] Tests, validation et déploiement
    \item[Chapitre 5] Résultats, conclusion et perspectives
\end{description}

Chaque chapitre détaille une phase spécifique du projet, depuis l'analyse initiale jusqu'à la mise en production, en passant par la conception et le développement. Des annexes techniques complètent ce rapport avec des détails sur l'architecture, les modèles de données et les procédures de déploiement.

\section{Méthodologie de travail}
Le projet a été mené selon une approche agile, avec des itérations courtes permettant d'ajuster régulièrement les fonctionnalités en fonction des retours des utilisateurs. Cette méthodologie a permis de :

\begin{itemize}
    \item Maintenir un dialogue constant avec les utilisateurs finaux
    \item Adapter rapidement le système aux besoins évolutifs
    \item Identifier et corriger rapidement les problèmes
    \item Assurer une qualité progressive du produit final
\end{itemize}

La collaboration étroite avec l'équipe IT d'OCP et les utilisateurs métier a été un facteur clé de succès du projet, permettant de comprendre précisément les besoins et de les traduire en fonctionnalités techniques appropriées.
