\chapter{Analyse des Besoins et Conception du Système}

\section{Analyse des besoins fonctionnels}
\subsection{Acteurs du système}
Le système de gestion des astreintes OCP implique plusieurs catégories d'utilisateurs avec des rôles et permissions distincts :

\begin{description}
    \item[\textbf{Administrateur}] Accès complet au système, gestion des utilisateurs, sites, secteurs et services
    \item[\textbf{Chef de Site}] Gestion des secteurs et services de son site, vue d'ensemble des équipes
    \item[\textbf{Chef de Secteur}] Gestion des services de son secteur, vue des ingénieurs et collaborateurs
    \item[\textbf{Chef de Service}] Gestion de son équipe, planification des astreintes
    \item[\textbf{Ingénieur}] Consultation des plannings, gestion de ses disponibilités
    \item[\textbf{Collaborateur}] Consultation des plannings, notification des indisponibilités
\end{description}

\subsection{Fonctionnalités principales}
\subsubsection{Gestion des utilisateurs}
\begin{itemize}
    \item Création et modification des comptes utilisateurs
    \item Attribution des rôles et permissions
    \item Gestion des profils et informations personnelles
    \item Authentification sécurisée avec JWT
\end{itemize}

\subsubsection{Gestion organisationnelle}
\begin{itemize}
    \item Création et gestion des sites industriels
    \item Organisation hiérarchique : Sites → Secteurs → Services
    \item Gestion des secteurs (Traitement, Extraction, Maintenance, Logistique, Qualité)
    \item Gestion des services au sein de chaque secteur
\end{itemize}

\subsubsection{Gestion des astreintes}
\begin{itemize}
    \item Création et modification des plannings d'astreinte
    \item Gestion des disponibilités et indisponibilités
    \item Système d'escalade automatique
    \item Notifications et alertes en temps réel
\end{itemize}

\subsubsection{Tableaux de bord et rapports}
\begin{itemize}
    \item Vue d'ensemble des équipes par site/secteur/service
    \item Statistiques de couverture des astreintes
    \item Historique des modifications et actions
    \item Rapports d'activité et de performance
\end{itemize}

\section{Analyse des besoins non fonctionnels}
\subsection{Performance}
\begin{itemize}
    \item Temps de réponse < 2 secondes pour les opérations courantes
    \item Support de 100+ utilisateurs simultanés
    \item Gestion efficace des requêtes de base de données
\end{itemize}

\subsection{Sécurité}
\begin{itemize}
    \item Authentification forte avec tokens JWT
    \item Chiffrement des données sensibles
    \item Gestion fine des permissions par rôle
    \item Audit trail de toutes les actions
\end{itemize}

\subsection{Disponibilité}
\begin{itemize}
    \item Disponibilité 99.9\% (8.76 heures d'indisponibilité par an)
    \item Sauvegarde automatique des données
    \item Procédures de récupération en cas de panne
\end{itemize}

\subsection{Maintenabilité}
\begin{itemize}
    \item Architecture modulaire et extensible
    \item Code documenté et suivant les bonnes pratiques
    \item Tests automatisés pour les fonctionnalités critiques
    \item Documentation technique complète
\end{itemize}

\section{Architecture générale du système}
\subsection{Architecture en couches}
Le système suit une architecture en trois couches principales :

\begin{figure}[h]
\centering
\includegraphics[width=0.8\textwidth]{images/architecture-generale.png}
\caption{Architecture générale du système OCP Astreinte}
\label{fig:architecture-generale}
\end{figure}

\subsubsection{Couche présentation (Frontend)}
\begin{itemize}
    \item Interface utilisateur React avec TypeScript
    \item Composants modulaires et réutilisables
    \item Design responsive avec Tailwind CSS
    \item Gestion d'état locale avec React Hooks
\end{itemize}

\subsubsection{Couche logique métier (Backend)}
\begin{itemize}
    \item API REST avec Node.js et Express
    \item Contrôleurs pour la logique métier
    \item Middleware d'authentification et d'autorisation
    \item Validation des données avec Joi
\end{itemize}

\subsubsection{Couche données (Base de données)}
\begin{itemize}
    \item Base de données MongoDB
    \item Modèles Mongoose pour la validation
    \item Indexation optimisée pour les requêtes fréquentes
    \item Sauvegarde et réplication
\end{itemize}

\section{Modèles de données}
\subsection{Modèle utilisateur}
\begin{figure}[h]
\centering
\includegraphics[width=0.7\textwidth]{images/modele-utilisateur.png}
\caption{Modèle de données utilisateur}
\label{fig:modele-utilisateur}
\end{figure}

Le modèle utilisateur inclut :
\begin{itemize}
    \item Informations d'identification (email, mot de passe hashé)
    \item Informations personnelles (nom, prénom, téléphone)
    \item Rôle et permissions dans le système
    \item Affectation organisationnelle (site, secteur, service)
    \item Préférences et paramètres personnels
\end{itemize}

\subsection{Modèle organisationnel}
\begin{figure}[h]
\centering
\includegraphics[width=0.8\textwidth]{images/modele-organisationnel.png}
\caption{Modèle de données organisationnel}
\label{fig:modele-organisationnel}
\end{figure}

L'organisation suit une hiérarchie stricte :
\begin{itemize}
    \item \textbf{Sites} : Unités géographiques (ex: Khouribga, Youssoufia)
    \item \textbf{Secteurs} : Divisions fonctionnelles (ex: Traitement, Maintenance)
    \item \textbf{Services} : Équipes opérationnelles au sein des secteurs
\end{itemize}

\subsection{Modèle des astreintes}
\begin{figure}[h]
\centering
\includegraphics[width=0.7\textwidth]{images/modele-astreinte.png}
\caption{Modèle de données des astreintes}
\label{fig:modele-astreinte}
\end{figure}

Le système de gestion des astreintes comprend :
\begin{itemize}
    \item Planning des astreintes par service
    \item Gestion des disponibilités individuelles
    \item Système d'escalade en cas d'indisponibilité
    \item Historique et traçabilité des modifications
\end{itemize}

\section{Interfaces utilisateur}
\subsection{Principes de design}
\begin{itemize}
    \item \textbf{Simplicité} : Interface claire et intuitive
    \item \textbf{Consistance} : Cohérence visuelle et comportementale
    \item \textbf{Efficacité} : Accès rapide aux fonctionnalités principales
    \item \textbf{Responsivité} : Adaptation aux différents écrans
\end{itemize}

\subsection{Pages principales}
\subsubsection{Dashboard principal}
\begin{itemize}
    \item Vue d'ensemble des sites et secteurs
    \item Statistiques en temps réel
    \item Accès rapide aux fonctionnalités principales
    \item Notifications et alertes
\end{itemize}

\subsubsection{Gestion des secteurs}
\begin{itemize}
    \item Liste des secteurs avec statistiques
    \item Création et modification des secteurs
    \item Gestion des services associés
    \item Vue des équipes par secteur
\end{itemize}

\subsubsection{Gestion des services}
\begin{itemize}
    \item Liste des services par secteur
    \item Gestion des équipes et des astreintes
    \item Planification des plannings
    \item Suivi des disponibilités
\end{itemize}

\subsubsection{Mon secteur/Mon service}
\begin{itemize}
    \item Vue personnalisée selon le rôle
    \item Gestion des équipes sous responsabilité
    \item Consultation des plannings
    \item Actions rapides
\end{itemize}

\section{Sécurité et permissions}
\subsection{Système d'authentification}
\begin{itemize}
    \item Connexion sécurisée avec email/mot de passe
    \item Génération de tokens JWT avec expiration
    \item Refresh automatique des tokens
    \item Déconnexion sécurisée
\end{itemize}

\subsection{Gestion des permissions}
\begin{itemize}
    \item Permissions basées sur les rôles (RBAC)
    \item Contrôle d'accès aux ressources
    \item Filtrage automatique des données selon le contexte
    \item Audit des actions utilisateur
\end{itemize}

\subsection{Sécurité des données}
\begin{itemize}
    \item Chiffrement des mots de passe avec bcrypt
    \item Validation et sanitisation des entrées
    \item Protection contre les injections et attaques XSS
    \item Headers de sécurité avec Helmet
\end{itemize}

\section{Intégrations et API}
\subsection{API REST}
Le système expose une API REST complète pour :
\begin{itemize}
    \item Gestion des utilisateurs et authentification
    \item Gestion des sites, secteurs et services
    \item Gestion des astreintes et plannings
    \item Consultation des statistiques et rapports
\end{itemize}

\subsection{Format des réponses}
\begin{itemize}
    \item Réponses JSON standardisées
    \item Codes de statut HTTP appropriés
    \item Gestion d'erreurs détaillée
    \item Pagination pour les listes volumineuses
\end{itemize}

\subsection{Documentation API}
\begin{itemize}
    \item Documentation interactive avec exemples
    \item Schémas de validation des données
    \item Codes d'erreur et messages
    \item Exemples de requêtes et réponses
\end{itemize}

\section{Considérations techniques}
\subsection{Technologies choisies}
\begin{itemize}
    \item \textbf{Frontend} : React 19, TypeScript, Tailwind CSS
    \item \textbf{Backend} : Node.js, Express, MongoDB
    \item \textbf{Base de données} : MongoDB avec Mongoose
    \item \textbf{Authentification} : JWT, bcrypt
\end{itemize}

\subsection{Justification des choix}
\begin{itemize}
    \item \textbf{React} : Écosystème mature, composants réutilisables
    \item \textbf{Node.js} : Performance, écosystème npm riche
    \item \textbf{MongoDB} : Flexibilité du schéma, performance
    \item \textbf{TypeScript} : Sécurité des types, meilleure maintenabilité
\end{itemize}

\subsection{Alternatives considérées}
\begin{itemize}
    \item \textbf{Frontend} : Vue.js, Angular (moins adaptés aux besoins)
    \item \textbf{Backend} : Python/Django, Java/Spring (complexité accrue)
    \item \textbf{Base de données} : PostgreSQL, MySQL (rigidité du schéma)
\end{itemize}

Cette phase de conception a permis de définir précisément les besoins du système et d'établir une architecture robuste et évolutive, adaptée aux contraintes opérationnelles d'OCP.
