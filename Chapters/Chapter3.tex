\chapter{Implémentation Technique et Développement}

\section{Environnement de développement}
\subsection{Configuration de l'environnement}
Le développement du système OCP Astreinte s'est déroulé dans un environnement technique moderne et standardisé :

\begin{itemize}
    \item \textbf{Système d'exploitation} : Windows 10/11, Linux Ubuntu
    \item \textbf{Éditeur de code} : Visual Studio Code avec extensions spécialisées
    \item \textbf{Versioning} : Git avec GitHub pour la gestion des versions
    \item \textbf{Base de données} : MongoDB Atlas (cloud) et MongoDB local pour le développement
    \item \textbf{API Testing} : Postman pour les tests d'API
\end{itemize}

\subsection{Outils de développement}
\begin{itemize}
    \item \textbf{Node.js} : Version 18+ pour le backend
    \item \textbf{npm} : Gestionnaire de paquets pour les dépendances
    \item \textbf{ESLint} : Linting et formatage du code
    \item \textbf{Prettier} : Formatage automatique du code
    \item \textbf{TypeScript} : Compilateur et vérificateur de types
\end{itemize}

\section{Architecture technique détaillée}
\subsection{Structure du projet}
Le projet est organisé en deux parties principales avec une architecture claire :

\begin{figure}[h]
\centering
\includegraphics[width=0.9\textwidth]{images/structure-projet.png}
\caption{Structure détaillée du projet OCP Astreinte}
\label{fig:structure-projet}
\end{figure}

\subsubsection{Frontend (dossier \texttt{frontend/})}
\begin{itemize}
    \item \texttt{src/pages/} : Pages principales de l'application
    \item \texttt{src/components/} : Composants réutilisables
    \item \texttt{src/services/} : Services API et utilitaires
    \item \texttt{src/types/} : Définitions TypeScript
    \item \texttt{src/utils/} : Fonctions utilitaires
\end{itemize}

\subsubsection{Backend (dossier \texttt{server/})}
\begin{itemize}
    \item \texttt{controllers/} : Logique métier des endpoints
    \item \texttt{models/} : Modèles de données Mongoose
    \item \texttt{routes/} : Définition des routes API
    \item \texttt{middleware/} : Middleware d'authentification et validation
    \item \texttt{config/} : Configuration de l'application
\end{itemize}

\section{Implémentation du frontend}
\subsection{Technologies utilisées}
Le frontend utilise un stack moderne et performant :

\begin{itemize}
    \item \textbf{React 19} : Framework principal avec hooks et composants fonctionnels
    \item \textbf{TypeScript} : Typage statique pour la robustesse du code
    \item \textbf{Tailwind CSS} : Framework CSS utilitaire pour le design
    \item \textbf{React Router} : Gestion de la navigation et du routage
    \item \textbf{Axios} : Client HTTP pour les appels API
    \textbf{React Hook Form} : Gestion des formulaires avec validation
\end{itemize}

\subsection{Architecture des composants}
\subsubsection{Composants de base}
\begin{figure}[h]
\centering
\includegraphics[width=0.8\textwidth]{images/composants-frontend.png}
\caption{Architecture des composants frontend}
\label{fig:composants-frontend}
\end{figure}

Les composants sont organisés selon une hiérarchie claire :
\begin{itemize}
    \item \textbf{Layout} : Structure générale de l'application
    \item \textbf{Pages} : Pages principales avec logique métier
    \item \textbf{Components} : Composants réutilisables
    \item \textbf{Forms} : Formulaires spécialisés
\end{itemize}

\subsubsection{Gestion d'état}
L'état de l'application est géré localement avec React Hooks :
\begin{itemize}
    \item \texttt{useState} : État local des composants
    \item \texttt{useEffect} : Effets de bord et appels API
    \item \texttt{useContext} : Partage d'état global (authentification)
    \item \texttt{useCallback} : Optimisation des performances
\end{itemize}

\subsection{Pages principales implémentées}
\subsubsection{Dashboard principal}
\begin{figure}[h]
\centering
\includegraphics[width=0.8\textwidth]{images/dashboard-principal.png}
\caption{Interface du dashboard principal}
\label{fig:dashboard-principal}
\end{figure}

Le dashboard principal (\texttt{src/pages/Dashboard/Dashboard.tsx}) offre :
\begin{itemize}
    \item Vue d'ensemble des sites et secteurs
    \item Statistiques en temps réel
    \item Navigation rapide vers les fonctionnalités
    \item Notifications et alertes
\end{itemize}

\subsubsection{Gestion des secteurs}
La page de gestion des secteurs (\texttt{src/pages/Admin/SecteurPage.tsx}) permet :
\begin{itemize}
    \item Affichage de la liste des secteurs avec statistiques
    \item Création de nouveaux secteurs
    \item Modification des secteurs existants
    \item Suppression logique (soft delete)
    \item Consultation des employés par secteur
\end{itemize}

\subsubsection{Gestion des services}
La page de gestion des services (\texttt{src/pages/Admin/ServicePage.tsx}) offre :
\begin{itemize}
    \item Gestion des services par secteur
    \item Création et modification des services
    \item Attribution des chefs de service
    \item Configuration du personnel minimum
\end{itemize}

\subsubsection{Vues spécialisées}
\begin{itemize}
    \item \texttt{MonSecteur.tsx} : Vue du chef de secteur
    \item \texttt{MesIngenieurs.tsx} : Gestion des ingénieurs
    \item \texttt{MonEquipe.tsx} : Vue du chef de service
\end{itemize}

\section{Implémentation du backend}
\subsection{Technologies utilisées}
Le backend est construit avec des technologies robustes et éprouvées :

\begin{itemize}
    \item \textbf{Node.js} : Runtime JavaScript côté serveur
    \item \textbf{Express} : Framework web pour l'API REST
    \item \textbf{MongoDB} : Base de données NoSQL
    \item \textbf{Mongoose} : ODM pour MongoDB
    \item \textbf{JWT} : Authentification par tokens
    \item \textbf{bcrypt} : Chiffrement des mots de passe
\end{itemize}

\subsection{Architecture de l'API}
\subsubsection{Structure des routes}
\begin{figure}[h]
\centering
\includegraphics[width=0.8\textwidth]{images/architecture-api.png}
\caption{Architecture de l'API REST}
\label{fig:architecture-api}
\end{figure}

L'API est organisée en modules logiques :
\begin{itemize}
    \item \texttt{/api/auth} : Authentification et gestion des utilisateurs
    \item \texttt{/api/org} : Gestion organisationnelle (sites, secteurs, services)
    \item \texttt{/api/users} : Gestion des utilisateurs avec filtres
    \item \texttt{/api/astreintes} : Gestion des astreintes et plannings
\end{itemize}

\subsubsection{Middleware d'authentification}
Le système d'authentification utilise JWT avec plusieurs niveaux de sécurité :

\begin{lstlisting}[language=JavaScript, caption=Middleware d'authentification]
const authenticateToken = async (req, res, next) => {
  try {
    const token = req.cookies.token || req.headers.authorization?.split(' ')[1];
    
    if (!token) {
      return res.status(401).json({ 
        success: false, 
        message: 'Token d\'authentification requis' 
      });
    }
    
    const decoded = jwt.verify(token, process.env.JWT_SECRET);
    req.user = decoded;
    next();
  } catch (error) {
    return res.status(403).json({ 
      success: false, 
      message: 'Token invalide ou expiré' 
    });
  }
};
\end{lstlisting}

\subsection{Modèles de données}
\subsubsection{Modèle utilisateur}
\begin{figure}[h]
\centering
\includegraphics[width=0.7\textwidth]{images/schema-utilisateur.png}
\caption{Schéma de données utilisateur}
\label{fig:schema-utilisateur}
\end{figure}

Le modèle utilisateur (\texttt{server/models/User.js}) définit :
\begin{itemize}
    \item Informations d'identification (email, mot de passe)
    \item Informations personnelles (nom, prénom, téléphone)
    \item Rôle et permissions (admin, chef\_site, chef\_secteur, etc.)
    \item Affectation organisationnelle (site, secteur, service)
    \item Métadonnées (dates de création/modification)
\end{itemize}

\subsubsection{Modèle organisationnel}
\begin{itemize}
    \item \textbf{Site} : Unité géographique avec coordonnées
    \item \textbf{Secteur} : Division fonctionnelle au sein d'un site
    \item \textbf{Service} : Équipe opérationnelle au sein d'un secteur
\end{itemize}

\subsection{Contrôleurs et logique métier}
\subsubsection{Contrôleur utilisateur}
Le contrôleur utilisateur (\texttt{server/controllers/userController.js}) gère :

\begin{lstlisting}[language=JavaScript, caption=Fonction de récupération des utilisateurs]
const getUsers = async (req, res) => {
  try {
    const { secteur, service, role, limit = 50 } = req.query;
    const filter = {};
    
    // Filtrage par secteur
    if (secteur) filter.secteur = secteur;
    
    // Filtrage par service
    if (service) filter.service = service;
    
    // Filtrage par rôle
    if (role) filter.role = role;
    
    // Filtrage automatique selon le rôle de l'utilisateur connecté
    const currentUser = req.user;
    if (currentUser) {
      if (currentUser.role === 'chef_secteur' && currentUser.secteur) {
        filter.secteur = currentUser.secteur;
        // Restriction des rôles visibles
        if (!role) filter.role = { $in: ['ingenieur', 'collaborateur'] };
      }
    }
    
    const users = await User.find(filter).limit(parseInt(limit));
    res.json({ success: true, data: users });
  } catch (error) {
    res.status(500).json({ success: false, message: error.message });
  }
};
\end{lstlisting}

\subsubsection{Contrôleur organisationnel}
Le contrôleur organisationnel gère les sites, secteurs et services avec :
\begin{itemize}
    \item CRUD complet pour chaque entité
    \item Validation des données avec Joi
    \item Gestion des dépendances et contraintes
    \item Statistiques et métriques
\end{itemize}

\section{Gestion de la sécurité}
\subsection{Authentification et autorisation}
\subsubsection{Système JWT}
\begin{itemize}
    \item Génération de tokens avec expiration (24h)
    \item Refresh automatique des tokens
    \item Stockage sécurisé dans les cookies HTTP-only
    \item Invalidation des tokens lors de la déconnexion
\end{itemize}

\subsubsection{Gestion des permissions}
Le système implémente un contrôle d'accès basé sur les rôles (RBAC) :
\begin{itemize}
    \item \textbf{Admin} : Accès complet au système
    \item \textbf{Chef de site} : Gestion de son site uniquement
    \item \textbf{Chef de secteur} : Gestion de son secteur uniquement
    \item \textbf{Chef de service} : Gestion de son service uniquement
    \item \textbf{Utilisateur} : Consultation limitée selon son affectation
\end{itemize}

\subsection{Sécurité des données}
\subsubsection{Validation des entrées}
\begin{itemize}
    \item Validation côté serveur avec Joi
    \item Sanitisation des données
    \item Protection contre les injections
    \item Validation des types et formats
\end{itemize}

\subsubsection{Protection contre les attaques}
\begin{itemize}
    \item Headers de sécurité avec Helmet
    \item Protection CSRF
    \item Rate limiting pour prévenir les attaques par force brute
    \item Validation des tokens et sessions
\end{itemize}

\section{Base de données et persistance}
\subsection{Configuration MongoDB}
\begin{itemize}
    \item \textbf{Atlas} : Base de données cloud pour la production
    \item \textbf{Local} : Base de données locale pour le développement
    \item \textbf{Indexation} : Index optimisés pour les requêtes fréquentes
    \item \textbf{Backup} : Sauvegarde automatique quotidienne
\end{itemize}

\subsection{Modèles Mongoose}
Les modèles Mongoose assurent :
\begin{itemize}
    \item Validation des schémas
    \item Hooks et middleware
    \item Relations entre entités
    \item Indexation automatique
\end{itemize}

\section{Intégration et déploiement}
\subsection{Configuration de l'environnement}
\begin{itemize}
    \item Variables d'environnement avec dotenv
    \item Configuration différenciée dev/prod
    \item Gestion des secrets et clés API
    \item Configuration de la base de données
\end{itemize}

\subsection{Scripts de déploiement}
\begin{itemize}
    \item Scripts npm pour le build et le démarrage
    \item Configuration PM2 pour la production
    \item Scripts de migration de base de données
    \item Procédures de rollback
\end{itemize}

\section{Tests et qualité du code}
\subsection{Tests unitaires}
\begin{itemize}
    \item Tests des composants React avec Jest
    \item Tests des contrôleurs backend
    \item Tests des modèles de données
    \item Couverture de code > 80\%
\end{itemize}

\subsection{Tests d'intégration}
\begin{itemize}
    \item Tests des endpoints API
    \item Tests des flux d'authentification
    \item Tests des scénarios métier
    \item Tests de performance
\end{itemize}

\subsection{Qualité du code}
\begin{itemize}
    \item Linting avec ESLint
    \item Formatage avec Prettier
    \item Vérification des types TypeScript
    \item Documentation du code
\end{itemize}

Cette phase d'implémentation a permis de transformer la conception en un système fonctionnel, robuste et sécurisé, respectant les bonnes pratiques de développement et les exigences métier d'OCP.
